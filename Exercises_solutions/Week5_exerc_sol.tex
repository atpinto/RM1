% Options for packages loaded elsewhere
\PassOptionsToPackage{unicode}{hyperref}
\PassOptionsToPackage{hyphens}{url}
\PassOptionsToPackage{dvipsnames,svgnames,x11names}{xcolor}
%
\documentclass[
  letterpaper,
  DIV=11,
  numbers=noendperiod]{scrreprt}

\usepackage{amsmath,amssymb}
\usepackage{iftex}
\ifPDFTeX
  \usepackage[T1]{fontenc}
  \usepackage[utf8]{inputenc}
  \usepackage{textcomp} % provide euro and other symbols
\else % if luatex or xetex
  \usepackage{unicode-math}
  \defaultfontfeatures{Scale=MatchLowercase}
  \defaultfontfeatures[\rmfamily]{Ligatures=TeX,Scale=1}
\fi
\usepackage{lmodern}
\ifPDFTeX\else  
    % xetex/luatex font selection
\fi
% Use upquote if available, for straight quotes in verbatim environments
\IfFileExists{upquote.sty}{\usepackage{upquote}}{}
\IfFileExists{microtype.sty}{% use microtype if available
  \usepackage[]{microtype}
  \UseMicrotypeSet[protrusion]{basicmath} % disable protrusion for tt fonts
}{}
\makeatletter
\@ifundefined{KOMAClassName}{% if non-KOMA class
  \IfFileExists{parskip.sty}{%
    \usepackage{parskip}
  }{% else
    \setlength{\parindent}{0pt}
    \setlength{\parskip}{6pt plus 2pt minus 1pt}}
}{% if KOMA class
  \KOMAoptions{parskip=half}}
\makeatother
\usepackage{xcolor}
\setlength{\emergencystretch}{3em} % prevent overfull lines
\setcounter{secnumdepth}{-\maxdimen} % remove section numbering
% Make \paragraph and \subparagraph free-standing
\makeatletter
\ifx\paragraph\undefined\else
  \let\oldparagraph\paragraph
  \renewcommand{\paragraph}{
    \@ifstar
      \xxxParagraphStar
      \xxxParagraphNoStar
  }
  \newcommand{\xxxParagraphStar}[1]{\oldparagraph*{#1}\mbox{}}
  \newcommand{\xxxParagraphNoStar}[1]{\oldparagraph{#1}\mbox{}}
\fi
\ifx\subparagraph\undefined\else
  \let\oldsubparagraph\subparagraph
  \renewcommand{\subparagraph}{
    \@ifstar
      \xxxSubParagraphStar
      \xxxSubParagraphNoStar
  }
  \newcommand{\xxxSubParagraphStar}[1]{\oldsubparagraph*{#1}\mbox{}}
  \newcommand{\xxxSubParagraphNoStar}[1]{\oldsubparagraph{#1}\mbox{}}
\fi
\makeatother

\usepackage{color}
\usepackage{fancyvrb}
\newcommand{\VerbBar}{|}
\newcommand{\VERB}{\Verb[commandchars=\\\{\}]}
\DefineVerbatimEnvironment{Highlighting}{Verbatim}{commandchars=\\\{\}}
% Add ',fontsize=\small' for more characters per line
\usepackage{framed}
\definecolor{shadecolor}{RGB}{241,243,245}
\newenvironment{Shaded}{\begin{snugshade}}{\end{snugshade}}
\newcommand{\AlertTok}[1]{\textcolor[rgb]{0.68,0.00,0.00}{#1}}
\newcommand{\AnnotationTok}[1]{\textcolor[rgb]{0.37,0.37,0.37}{#1}}
\newcommand{\AttributeTok}[1]{\textcolor[rgb]{0.40,0.45,0.13}{#1}}
\newcommand{\BaseNTok}[1]{\textcolor[rgb]{0.68,0.00,0.00}{#1}}
\newcommand{\BuiltInTok}[1]{\textcolor[rgb]{0.00,0.23,0.31}{#1}}
\newcommand{\CharTok}[1]{\textcolor[rgb]{0.13,0.47,0.30}{#1}}
\newcommand{\CommentTok}[1]{\textcolor[rgb]{0.37,0.37,0.37}{#1}}
\newcommand{\CommentVarTok}[1]{\textcolor[rgb]{0.37,0.37,0.37}{\textit{#1}}}
\newcommand{\ConstantTok}[1]{\textcolor[rgb]{0.56,0.35,0.01}{#1}}
\newcommand{\ControlFlowTok}[1]{\textcolor[rgb]{0.00,0.23,0.31}{\textbf{#1}}}
\newcommand{\DataTypeTok}[1]{\textcolor[rgb]{0.68,0.00,0.00}{#1}}
\newcommand{\DecValTok}[1]{\textcolor[rgb]{0.68,0.00,0.00}{#1}}
\newcommand{\DocumentationTok}[1]{\textcolor[rgb]{0.37,0.37,0.37}{\textit{#1}}}
\newcommand{\ErrorTok}[1]{\textcolor[rgb]{0.68,0.00,0.00}{#1}}
\newcommand{\ExtensionTok}[1]{\textcolor[rgb]{0.00,0.23,0.31}{#1}}
\newcommand{\FloatTok}[1]{\textcolor[rgb]{0.68,0.00,0.00}{#1}}
\newcommand{\FunctionTok}[1]{\textcolor[rgb]{0.28,0.35,0.67}{#1}}
\newcommand{\ImportTok}[1]{\textcolor[rgb]{0.00,0.46,0.62}{#1}}
\newcommand{\InformationTok}[1]{\textcolor[rgb]{0.37,0.37,0.37}{#1}}
\newcommand{\KeywordTok}[1]{\textcolor[rgb]{0.00,0.23,0.31}{\textbf{#1}}}
\newcommand{\NormalTok}[1]{\textcolor[rgb]{0.00,0.23,0.31}{#1}}
\newcommand{\OperatorTok}[1]{\textcolor[rgb]{0.37,0.37,0.37}{#1}}
\newcommand{\OtherTok}[1]{\textcolor[rgb]{0.00,0.23,0.31}{#1}}
\newcommand{\PreprocessorTok}[1]{\textcolor[rgb]{0.68,0.00,0.00}{#1}}
\newcommand{\RegionMarkerTok}[1]{\textcolor[rgb]{0.00,0.23,0.31}{#1}}
\newcommand{\SpecialCharTok}[1]{\textcolor[rgb]{0.37,0.37,0.37}{#1}}
\newcommand{\SpecialStringTok}[1]{\textcolor[rgb]{0.13,0.47,0.30}{#1}}
\newcommand{\StringTok}[1]{\textcolor[rgb]{0.13,0.47,0.30}{#1}}
\newcommand{\VariableTok}[1]{\textcolor[rgb]{0.07,0.07,0.07}{#1}}
\newcommand{\VerbatimStringTok}[1]{\textcolor[rgb]{0.13,0.47,0.30}{#1}}
\newcommand{\WarningTok}[1]{\textcolor[rgb]{0.37,0.37,0.37}{\textit{#1}}}

\providecommand{\tightlist}{%
  \setlength{\itemsep}{0pt}\setlength{\parskip}{0pt}}\usepackage{longtable,booktabs,array}
\usepackage{calc} % for calculating minipage widths
% Correct order of tables after \paragraph or \subparagraph
\usepackage{etoolbox}
\makeatletter
\patchcmd\longtable{\par}{\if@noskipsec\mbox{}\fi\par}{}{}
\makeatother
% Allow footnotes in longtable head/foot
\IfFileExists{footnotehyper.sty}{\usepackage{footnotehyper}}{\usepackage{footnote}}
\makesavenoteenv{longtable}
\usepackage{graphicx}
\makeatletter
\newsavebox\pandoc@box
\newcommand*\pandocbounded[1]{% scales image to fit in text height/width
  \sbox\pandoc@box{#1}%
  \Gscale@div\@tempa{\textheight}{\dimexpr\ht\pandoc@box+\dp\pandoc@box\relax}%
  \Gscale@div\@tempb{\linewidth}{\wd\pandoc@box}%
  \ifdim\@tempb\p@<\@tempa\p@\let\@tempa\@tempb\fi% select the smaller of both
  \ifdim\@tempa\p@<\p@\scalebox{\@tempa}{\usebox\pandoc@box}%
  \else\usebox{\pandoc@box}%
  \fi%
}
% Set default figure placement to htbp
\def\fps@figure{htbp}
\makeatother

\KOMAoption{captions}{tableheading}
\makeatletter
\@ifpackageloaded{caption}{}{\usepackage{caption}}
\AtBeginDocument{%
\ifdefined\contentsname
  \renewcommand*\contentsname{Table of contents}
\else
  \newcommand\contentsname{Table of contents}
\fi
\ifdefined\listfigurename
  \renewcommand*\listfigurename{List of Figures}
\else
  \newcommand\listfigurename{List of Figures}
\fi
\ifdefined\listtablename
  \renewcommand*\listtablename{List of Tables}
\else
  \newcommand\listtablename{List of Tables}
\fi
\ifdefined\figurename
  \renewcommand*\figurename{Figure}
\else
  \newcommand\figurename{Figure}
\fi
\ifdefined\tablename
  \renewcommand*\tablename{Table}
\else
  \newcommand\tablename{Table}
\fi
}
\@ifpackageloaded{float}{}{\usepackage{float}}
\floatstyle{ruled}
\@ifundefined{c@chapter}{\newfloat{codelisting}{h}{lop}}{\newfloat{codelisting}{h}{lop}[chapter]}
\floatname{codelisting}{Listing}
\newcommand*\listoflistings{\listof{codelisting}{List of Listings}}
\makeatother
\makeatletter
\makeatother
\makeatletter
\@ifpackageloaded{caption}{}{\usepackage{caption}}
\@ifpackageloaded{subcaption}{}{\usepackage{subcaption}}
\makeatother

\usepackage{bookmark}

\IfFileExists{xurl.sty}{\usepackage{xurl}}{} % add URL line breaks if available
\urlstyle{same} % disable monospaced font for URLs
\hypersetup{
  pdftitle={Week5-Exercises-Solutions},
  colorlinks=true,
  linkcolor={blue},
  filecolor={Maroon},
  citecolor={Blue},
  urlcolor={Blue},
  pdfcreator={LaTeX via pandoc}}


\title{Week5-Exercises-Solutions}
\author{}
\date{}

\begin{document}
\maketitle


\chapter{Exercise solutions}\label{exercise-solutions}

\subsection*{Week 5}\label{week-5}
\addcontentsline{toc}{subsection}{Week 5}

\subsubsection*{Exercise 1}\label{exercise-1}
\addcontentsline{toc}{subsubsection}{Exercise 1}

The solutions are embedded in the notes.

\subsection{\texorpdfstring{\textbf{Solution in
R}}{Solution in R}}\label{solution-in-r}

\subsubsection*{Exercise 2.1 Reproduce the adjusted analysis of glucose
carried out in p.~72. Make sure that you exclude diabetic
patients}\label{exercise-2.1-reproduce-the-adjusted-analysis-of-glucose-carried-out-in-p.-72.-make-sure-that-you-exclude-diabetic-patients}
\addcontentsline{toc}{subsubsection}{Exercise 2.1 Reproduce the adjusted
analysis of glucose carried out in p.~72. Make sure that you exclude
diabetic patients}

The objective is to reproduce the adjusted analysis of Vittinghof et
al.~(2012) for glucose in non-diabetic patients given p.~72.

First, we select the right dataset and then use \emph{lm} with the
proper model to get the results we seek:

\begin{Shaded}
\begin{Highlighting}[]

\NormalTok{hers }\OtherTok{\textless{}{-}} \FunctionTok{read.csv}\NormalTok{(}\StringTok{"https://www.dropbox.com/scl/fi/ywlbb7duvez2nyk66ojp1/hersdata.csv?rlkey=tmhzlv6ga5zp6uyysnosaqamj\&st=gl7h7ym8\&dl=1"}\NormalTok{,             }\AttributeTok{stringsAsFactors =}\NormalTok{ T) }
\NormalTok{hers.nondiab }\OtherTok{\textless{}{-}}\NormalTok{ hers[hers}\SpecialCharTok{$}\NormalTok{diabetes}\SpecialCharTok{==}\StringTok{"no"}\NormalTok{,]}



\CommentTok{\#deletes two missing value for the variable drinkany}
\NormalTok{hers.nondiab}\SpecialCharTok{$}\NormalTok{drinkany[hers.nondiab}\SpecialCharTok{$}\NormalTok{drinkany}\SpecialCharTok{==}\StringTok{""}\NormalTok{] }\OtherTok{\textless{}{-}} \ConstantTok{NA}
\NormalTok{hers.nondiab}\SpecialCharTok{$}\NormalTok{drinkany }\OtherTok{\textless{}{-}} \FunctionTok{droplevels}\NormalTok{(hers.nondiab}\SpecialCharTok{$}\NormalTok{drinkany)}


\NormalTok{fit }\OtherTok{\textless{}{-}} \FunctionTok{lm}\NormalTok{(glucose }\SpecialCharTok{\textasciitilde{}}\NormalTok{ exercise }\SpecialCharTok{+}\NormalTok{ age }\SpecialCharTok{+}\NormalTok{ drinkany }\SpecialCharTok{+}\NormalTok{ BMI, }\AttributeTok{data =}\NormalTok{ hers.nondiab)}
\FunctionTok{summary}\NormalTok{(fit)}
\DocumentationTok{\#\# }
\DocumentationTok{\#\# Call:}
\DocumentationTok{\#\# lm(formula = glucose \textasciitilde{} exercise + age + drinkany + BMI, data = hers.nondiab)}
\DocumentationTok{\#\# }
\DocumentationTok{\#\# Residuals:}
\DocumentationTok{\#\#     Min      1Q  Median      3Q     Max }
\DocumentationTok{\#\# {-}47.560  {-}6.400  {-}0.886   5.496  32.060 }
\DocumentationTok{\#\# }
\DocumentationTok{\#\# Coefficients:}
\DocumentationTok{\#\#             Estimate Std. Error t value Pr(\textgreater{}|t|)    }
\DocumentationTok{\#\# (Intercept) 78.96239    2.59284  30.454   \textless{}2e{-}16 ***}
\DocumentationTok{\#\# exerciseyes {-}0.95044    0.42873  {-}2.217   0.0267 *  }
\DocumentationTok{\#\# age          0.06355    0.03139   2.024   0.0431 *  }
\DocumentationTok{\#\# drinkanyyes  0.68026    0.42196   1.612   0.1071    }
\DocumentationTok{\#\# BMI          0.48924    0.04155  11.774   \textless{}2e{-}16 ***}
\DocumentationTok{\#\# {-}{-}{-}}
\DocumentationTok{\#\# Signif. codes:  0 \textquotesingle{}***\textquotesingle{} 0.001 \textquotesingle{}**\textquotesingle{} 0.01 \textquotesingle{}*\textquotesingle{} 0.05 \textquotesingle{}.\textquotesingle{} 0.1 \textquotesingle{} \textquotesingle{} 1}
\DocumentationTok{\#\# }
\DocumentationTok{\#\# Residual standard error: 9.389 on 2023 degrees of freedom}
\DocumentationTok{\#\#   (4 observations deleted due to missingness)}
\DocumentationTok{\#\# Multiple R{-}squared:  0.07197,    Adjusted R{-}squared:  0.07013 }
\DocumentationTok{\#\# F{-}statistic: 39.22 on 4 and 2023 DF,  p{-}value: \textless{} 2.2e{-}16}
\end{Highlighting}
\end{Shaded}

\subsubsection*{Exercise 2.2 Use matrix operations in Stata or R to
create the X,Y,X′X and X′Y matrices and use these to obtain the LS
estimates. {[}Caution: there are missing values in some of these
covariates so delete first all observations with missing values before
any matrix
manipulation{]}}\label{exercise-2.2-use-matrix-operations-in-stata-or-r-to-create-the-xyxx-and-xy-matrices-and-use-these-to-obtain-the-ls-estimates.-caution-there-are-missing-values-in-some-of-these-covariates-so-delete-first-all-observations-with-missing-values-before-any-matrix-manipulation}
\addcontentsline{toc}{subsubsection}{Exercise 2.2 Use matrix operations
in Stata or R to create the X,Y,X′X and X′Y matrices and use these to
obtain the LS estimates. {[}Caution: there are missing values in some of
these covariates so delete first all observations with missing values
before any matrix manipulation{]}}

Matrix manipulations similar to the ones carried out in Exercise 1 gives
us the LSE

\begin{Shaded}
\begin{Highlighting}[]
\CommentTok{\# creates the reduced dataset}
\NormalTok{hers.nondiab1 }\OtherTok{\textless{}{-}}\NormalTok{ hers.nondiab[, }\FunctionTok{c}\NormalTok{(}\StringTok{"glucose"}\NormalTok{, }
                      \StringTok{"exercise"}\NormalTok{, }\StringTok{"age"}\NormalTok{, }
                      \StringTok{"drinkany"}\NormalTok{, }\StringTok{"BMI"}\NormalTok{)]}


\NormalTok{hers.nondiab1 }\OtherTok{\textless{}{-}}\FunctionTok{na.omit}\NormalTok{(hers.nondiab1)}

\CommentTok{\# create the vector of responses and the design matrix}
\CommentTok{\# do not forget the columns of ones for the intercept}
\CommentTok{\# check the dimension of the objects that you created}



\NormalTok{Y }\OtherTok{\textless{}{-}}\NormalTok{ hers.nondiab1}\SpecialCharTok{$}\NormalTok{glucose}
\NormalTok{X }\OtherTok{\textless{}{-}} \FunctionTok{model.matrix}\NormalTok{(glucose }\SpecialCharTok{\textasciitilde{}}\NormalTok{ exercise }\SpecialCharTok{+}\NormalTok{ age }\SpecialCharTok{+}  \CommentTok{\#gives the design matrix}
\NormalTok{                    drinkany }\SpecialCharTok{+}\NormalTok{ BMI, }
                  \AttributeTok{data =}\NormalTok{ hers.nondiab1)}


\NormalTok{X }\OtherTok{\textless{}{-}} \FunctionTok{cbind}\NormalTok{(}\DecValTok{1}\NormalTok{, hers.nondiab1}\SpecialCharTok{$}\NormalTok{exercise, hers.nondiab1}\SpecialCharTok{$}\NormalTok{age, hers.nondiab1}\SpecialCharTok{$}\NormalTok{drinkany, hers.nondiab1}\SpecialCharTok{$}\NormalTok{BMI)}



\FunctionTok{dim}\NormalTok{(X)}
\DocumentationTok{\#\# [1] 2028    5}
\FunctionTok{length}\NormalTok{(Y)}
\DocumentationTok{\#\# [1] 2028}

\CommentTok{\# calculate the LS estimate (called b) and print it}
\NormalTok{XTX}\OtherTok{\textless{}{-}}\FunctionTok{t}\NormalTok{(X) }\SpecialCharTok{\%*\%}\NormalTok{ X}
\NormalTok{b}\OtherTok{=}\FunctionTok{solve}\NormalTok{(XTX)}\SpecialCharTok{\%*\%}\FunctionTok{t}\NormalTok{(X)}\SpecialCharTok{\%*\%}\NormalTok{Y}
\NormalTok{b}
\DocumentationTok{\#\#             [,1]}
\DocumentationTok{\#\# [1,] 79.23257076}
\DocumentationTok{\#\# [2,] {-}0.95044096}
\DocumentationTok{\#\# [3,]  0.06354948}
\DocumentationTok{\#\# [4,]  0.68026413}
\DocumentationTok{\#\# [5,]  0.48924198}
\end{Highlighting}
\end{Shaded}

\subsubsection*{Exercise 2.2 Optional: Use an explicit matrix
calculation in Stata/R to obtain the variance-covariance matrix for b in
the regression of glucose on the previous covariates. Calculate the
standard errors and confirm your results by comparing with the
regression
output.}\label{exercise-2.2-optional-use-an-explicit-matrix-calculation-in-statar-to-obtain-the-variance-covariance-matrix-for-b-in-the-regression-of-glucose-on-the-previous-covariates.-calculate-the-standard-errors-and-confirm-your-results-by-comparing-with-the-regression-output.}
\addcontentsline{toc}{subsubsection}{Exercise 2.2 Optional: Use an
explicit matrix calculation in Stata/R to obtain the variance-covariance
matrix for b in the regression of glucose on the previous covariates.
Calculate the standard errors and confirm your results by comparing with
the regression output.}

This is the same as what has been reported in the R output. You can get
the SEs using the following code.

\begin{Shaded}
\begin{Highlighting}[]

\NormalTok{sigma2 }\OtherTok{\textless{}{-}} \FunctionTok{sum}\NormalTok{((Y}\SpecialCharTok{{-}}\NormalTok{ X}\SpecialCharTok{\%*\%}\NormalTok{b)}\SpecialCharTok{\^{}}\DecValTok{2}\NormalTok{)}\SpecialCharTok{/}\NormalTok{(}\DecValTok{406{-}5}\NormalTok{) }\CommentTok{\#var of the residuals}
\NormalTok{matvar }\OtherTok{\textless{}{-}}\NormalTok{ sigma2}\SpecialCharTok{*}\FunctionTok{solve}\NormalTok{(XTX)     }\CommentTok{\#Var{-}Covariance matrix   }
\NormalTok{matvar}
\DocumentationTok{\#\#            [,1]          [,2]          [,3]         [,4]          [,5]}
\DocumentationTok{\#\# [1,] 39.2325167 {-}1.7605965826 {-}0.3698689805 {-}1.803669933 {-}0.3318138287}
\DocumentationTok{\#\# [2,] {-}1.7605966  0.9272978402  0.0009627852 {-}0.006771813  0.0142594095}
\DocumentationTok{\#\# [3,] {-}0.3698690  0.0009627852  0.0049712413  0.006216119  0.0009753379}
\DocumentationTok{\#\# [4,] {-}1.8036699 {-}0.0067718131  0.0062161192  0.898230415  0.0037299508}
\DocumentationTok{\#\# [5,] {-}0.3318138  0.0142594095  0.0009753379  0.003729951  0.0087106974}

\CommentTok{\# extract the diagonal and take the square root {-} to get the SEs}
\NormalTok{SE}\OtherTok{\textless{}{-}}\FunctionTok{sqrt}\NormalTok{(}\FunctionTok{diag}\NormalTok{(matvar))}
\NormalTok{SE}
\DocumentationTok{\#\# [1] 6.26358657 0.96296305 0.07050703 0.94775019 0.09333112}
\end{Highlighting}
\end{Shaded}

\subsubsection*{Exercise 3.1 Using your favourite software compute the
95\% CI for the mean glucose of a patient aged 65, who does not drink
nor exercise and has
BMI=29.}\label{exercise-3.1-using-your-favourite-software-compute-the-95-ci-for-the-mean-glucose-of-a-patient-aged-65-who-does-not-drink-nor-exercise-and-has-bmi29.}
\addcontentsline{toc}{subsubsection}{Exercise 3.1 Using your favourite
software compute the 95\% CI for the mean glucose of a patient aged 65,
who does not drink nor exercise and has BMI=29.}

\textbf{Solution in R}

The predicted value and its 95\% CI (for a particular patient profile)
can be directly obtained in R via the command predict. Note that the
option \emph{confidence} must be used for the confidence interval and
\emph{prediction} for a (wider) prediction interval.

\begin{Shaded}
\begin{Highlighting}[]
\NormalTok{fit }\OtherTok{\textless{}{-}} \FunctionTok{lm}\NormalTok{(glucose }\SpecialCharTok{\textasciitilde{}}\NormalTok{ exercise }\SpecialCharTok{+}\NormalTok{ age }\SpecialCharTok{+}\NormalTok{ drinkany }\SpecialCharTok{+}\NormalTok{ BMI, }\AttributeTok{data =}\NormalTok{ hers.nondiab1)}
\FunctionTok{summary}\NormalTok{(fit)}
\DocumentationTok{\#\# }
\DocumentationTok{\#\# Call:}
\DocumentationTok{\#\# lm(formula = glucose \textasciitilde{} exercise + age + drinkany + BMI, data = hers.nondiab1)}
\DocumentationTok{\#\# }
\DocumentationTok{\#\# Residuals:}
\DocumentationTok{\#\#     Min      1Q  Median      3Q     Max }
\DocumentationTok{\#\# {-}47.560  {-}6.400  {-}0.886   5.496  32.060 }
\DocumentationTok{\#\# }
\DocumentationTok{\#\# Coefficients:}
\DocumentationTok{\#\#             Estimate Std. Error t value Pr(\textgreater{}|t|)    }
\DocumentationTok{\#\# (Intercept) 78.96239    2.59284  30.454   \textless{}2e{-}16 ***}
\DocumentationTok{\#\# exerciseyes {-}0.95044    0.42873  {-}2.217   0.0267 *  }
\DocumentationTok{\#\# age          0.06355    0.03139   2.024   0.0431 *  }
\DocumentationTok{\#\# drinkanyyes  0.68026    0.42196   1.612   0.1071    }
\DocumentationTok{\#\# BMI          0.48924    0.04155  11.774   \textless{}2e{-}16 ***}
\DocumentationTok{\#\# {-}{-}{-}}
\DocumentationTok{\#\# Signif. codes:  0 \textquotesingle{}***\textquotesingle{} 0.001 \textquotesingle{}**\textquotesingle{} 0.01 \textquotesingle{}*\textquotesingle{} 0.05 \textquotesingle{}.\textquotesingle{} 0.1 \textquotesingle{} \textquotesingle{} 1}
\DocumentationTok{\#\# }
\DocumentationTok{\#\# Residual standard error: 9.389 on 2023 degrees of freedom}
\DocumentationTok{\#\# Multiple R{-}squared:  0.07197,    Adjusted R{-}squared:  0.07013 }
\DocumentationTok{\#\# F{-}statistic: 39.22 on 4 and 2023 DF,  p{-}value: \textless{} 2.2e{-}16}
\NormalTok{new.data }\OtherTok{\textless{}{-}} \FunctionTok{data.frame}\NormalTok{(}\AttributeTok{exercise=}\StringTok{"no"}\NormalTok{, }\AttributeTok{age =} \DecValTok{65}\NormalTok{, }\AttributeTok{drinkany=}\StringTok{"no"}\NormalTok{, }\AttributeTok{BMI=}\DecValTok{29}\NormalTok{)}
\DocumentationTok{\#\# 95\% CI for the mean }
\NormalTok{pred.mean }\OtherTok{\textless{}{-}} \FunctionTok{predict}\NormalTok{(fit, new.data, }\AttributeTok{interval =} \StringTok{"confidence"}\NormalTok{)}
\NormalTok{pred.mean}
\DocumentationTok{\#\#        fit      lwr      upr}
\DocumentationTok{\#\# 1 97.28113 96.62155 97.94071}
\CommentTok{\# 95\% prediction interval }
\NormalTok{pred.forecast }\OtherTok{\textless{}{-}} \FunctionTok{predict}\NormalTok{(fit, new.data, }\AttributeTok{interval =} \StringTok{"prediction"}\NormalTok{)}
\NormalTok{pred.forecast}
\DocumentationTok{\#\#        fit      lwr      upr}
\DocumentationTok{\#\# 1 97.28113 78.85693 115.7053}
\end{Highlighting}
\end{Shaded}

Can you reproduce this result using matrix manipulations and the formula
given above?

Algebraically it follows from (we assumed here that we still have the
LSE estimates \(b\) and the fitted model. If you want to use you
calculated estimate \(\hat\sigma\) it should work just the same.

\begin{Shaded}
\begin{Highlighting}[]
\NormalTok{x    }\OtherTok{\textless{}{-}} \FunctionTok{c}\NormalTok{(}\DecValTok{1}\NormalTok{,}\DecValTok{0}\NormalTok{,}\DecValTok{65}\NormalTok{,}\DecValTok{0}\NormalTok{,}\DecValTok{29}\NormalTok{)}
\NormalTok{pred.manual }\OtherTok{\textless{}{-}}\NormalTok{ x}\SpecialCharTok{\%*\%}\NormalTok{b}

\NormalTok{SE}\OtherTok{\textless{}{-}} \FunctionTok{summary}\NormalTok{(fit)}\SpecialCharTok{$}\NormalTok{sigma}\SpecialCharTok{*}\FunctionTok{sqrt}\NormalTok{(x}\SpecialCharTok{\%*\%}\FunctionTok{solve}\NormalTok{(}\FunctionTok{t}\NormalTok{(X) }\SpecialCharTok{\%*\%}\NormalTok{ X)}\SpecialCharTok{\%*\%}\FunctionTok{cbind}\NormalTok{(x))}
\NormalTok{lower}\OtherTok{\textless{}{-}}\NormalTok{pred.manual}\FloatTok{{-}1.96}\SpecialCharTok{*}\NormalTok{SE}
\NormalTok{upper}\OtherTok{\textless{}{-}}\NormalTok{pred.manual}\FloatTok{+1.96}\SpecialCharTok{*}\NormalTok{SE}
\FunctionTok{c}\NormalTok{(pred.manual,lower,upper)}
\DocumentationTok{\#\# [1] 97.55130 95.82567 99.27694}
\end{Highlighting}
\end{Shaded}

The same values (up to rounding) are obtained. We used here 1.96 since
the sample is large but the exact quantile from the appropriate
\(t\)-distribution should be used in smaller samples.

\subsection{\texorpdfstring{\textbf{Solution in
Stata}}{Solution in Stata}}\label{solution-in-stata}

NOTE: I will use the subset of the hers data as I am not able to make
calculations with matrices larger than 800 rows (Stata version
constrain!)

\subsubsection*{Exercise 2.1 Reproduce the adjusted analysis of glucose
carried out in p.~72. Make sure that you exclude diabetic
patients}\label{exercise-2.1-reproduce-the-adjusted-analysis-of-glucose-carried-out-in-p.-72.-make-sure-that-you-exclude-diabetic-patients-1}
\addcontentsline{toc}{subsubsection}{Exercise 2.1 Reproduce the adjusted
analysis of glucose carried out in p.~72. Make sure that you exclude
diabetic patients}

The objective is to reproduce the adjusted analysis of Vittinghof et
al.~(2012) for glucose in non-diabetic patients given p.~72.

First, we select the right dataset and then use \emph{regress} with the
proper model to get the results we seek:

\begin{Shaded}
\begin{Highlighting}[]
\NormalTok{* Stata code}

\KeywordTok{clear}
\NormalTok{import delimited }\StringTok{"https://www.dropbox.com/scl/fi/9dtsid3cpziubhhuw9hhy/hers\_subset.csv?rlkey=vainwt6vtbbo2kuqidv0e24v9\&st=k5r7e42w\&dl=1"}
\KeywordTok{encode}\NormalTok{  exercise, }\KeywordTok{gen}\NormalTok{(exercise\_r)}
\KeywordTok{encode}\NormalTok{ drinkany , }\KeywordTok{gen}\NormalTok{(drinkany\_r)}

\KeywordTok{drop} \KeywordTok{if}\NormalTok{ diabetes == }\StringTok{"yes"}

\NormalTok{* We will }\KeywordTok{use}\NormalTok{ only 20\% }\KeywordTok{of}\NormalTok{ the }\KeywordTok{data}
\NormalTok{* Due to the licence limitation }\KeywordTok{of} \KeywordTok{STATA}
\NormalTok{* Otherwise you will }\FunctionTok{get}\NormalTok{ an }\KeywordTok{error}\NormalTok{ message}

\KeywordTok{sample}\NormalTok{ 20}


\KeywordTok{regress}\NormalTok{ glucose i.exercise\_r age i.drinkany\_r bmi }

\NormalTok{\#\# (encoding automatically selected: ISO{-}8859{-}1)}
\NormalTok{\#\# (38 vars, 276 }\KeywordTok{obs}\NormalTok{)}
\NormalTok{\#\# }
\NormalTok{\#\# }
\NormalTok{\#\# }
\NormalTok{\#\# (85 observations deleted)}
\NormalTok{\#\# }
\NormalTok{\#\# (153 observations deleted)}
\NormalTok{\#\# }
\NormalTok{\#\# }
\NormalTok{\#\#       Source |       SS           df       MS      Number }\KeywordTok{of} \KeywordTok{obs}\NormalTok{   =        38}
\NormalTok{\#\# {-}{-}{-}{-}{-}{-}{-}{-}{-}{-}{-}{-}{-}+{-}{-}{-}{-}{-}{-}{-}{-}{-}{-}{-}{-}{-}{-}{-}{-}{-}{-}{-}{-}{-}{-}{-}{-}{-}{-}{-}{-}{-}{-}{-}{-}{-}{-}   }\FunctionTok{F}\NormalTok{(4, 33)        =      1.50}
\NormalTok{\#\#        Model |  625.724716         4  156.431179   Prob \textgreater{} }\FunctionTok{F}\NormalTok{        =    0.2252}
\NormalTok{\#\#     Residual |  3445.11739        33  104.397497   R{-}squared       =    0.1537}
\NormalTok{\#\# {-}{-}{-}{-}{-}{-}{-}{-}{-}{-}{-}{-}{-}+{-}{-}{-}{-}{-}{-}{-}{-}{-}{-}{-}{-}{-}{-}{-}{-}{-}{-}{-}{-}{-}{-}{-}{-}{-}{-}{-}{-}{-}{-}{-}{-}{-}{-}   Adj R{-}squared   =    0.0511}
\NormalTok{\#\#        Total |  4070.84211        37   110.02276   Root MSE        =    10.218}
\NormalTok{\#\# }
\NormalTok{\#\# {-}{-}{-}{-}{-}{-}{-}{-}{-}{-}{-}{-}{-}{-}{-}{-}{-}{-}{-}{-}{-}{-}{-}{-}{-}{-}{-}{-}{-}{-}{-}{-}{-}{-}{-}{-}{-}{-}{-}{-}{-}{-}{-}{-}{-}{-}{-}{-}{-}{-}{-}{-}{-}{-}{-}{-}{-}{-}{-}{-}{-}{-}{-}{-}{-}{-}{-}{-}{-}{-}{-}{-}{-}{-}{-}{-}{-}{-}}
\NormalTok{\#\#      glucose | Coefficient  Std. err.      t    P\textgreater{}|t|     [95\% conf. interval]}
\NormalTok{\#\# {-}{-}{-}{-}{-}{-}{-}{-}{-}{-}{-}{-}{-}+{-}{-}{-}{-}{-}{-}{-}{-}{-}{-}{-}{-}{-}{-}{-}{-}{-}{-}{-}{-}{-}{-}{-}{-}{-}{-}{-}{-}{-}{-}{-}{-}{-}{-}{-}{-}{-}{-}{-}{-}{-}{-}{-}{-}{-}{-}{-}{-}{-}{-}{-}{-}{-}{-}{-}{-}{-}{-}{-}{-}{-}{-}{-}{-}}
\NormalTok{\#\#   exercise\_r |}
\NormalTok{\#\#         yes  |  {-}.8196387   3.683401    {-}0.22   0.825    {-}8.313574    6.674297}
\NormalTok{\#\#          age |  {-}.3129571   .3041805    {-}1.03   0.311    {-}.9318169    .3059027}
\NormalTok{\#\#              |}
\NormalTok{\#\#   drinkany\_r |}
\NormalTok{\#\#         yes  |  {-}6.013101   3.734437    {-}1.61   0.117    {-}13.61087    1.584667}
\NormalTok{\#\#          bmi |   .3334854   .3522123     0.95   0.351    {-}.3830959    1.050067}
\NormalTok{\#\#        }\DataTypeTok{\_cons}\NormalTok{ |   110.6162   24.42434     4.53   0.000     60.92448    160.3079}
\NormalTok{\#\# {-}{-}{-}{-}{-}{-}{-}{-}{-}{-}{-}{-}{-}{-}{-}{-}{-}{-}{-}{-}{-}{-}{-}{-}{-}{-}{-}{-}{-}{-}{-}{-}{-}{-}{-}{-}{-}{-}{-}{-}{-}{-}{-}{-}{-}{-}{-}{-}{-}{-}{-}{-}{-}{-}{-}{-}{-}{-}{-}{-}{-}{-}{-}{-}{-}{-}{-}{-}{-}{-}{-}{-}{-}{-}{-}{-}{-}{-}}
\end{Highlighting}
\end{Shaded}

Matrix manipulations similar to the ones carried out in Exercise 1 gives
us the LSE (after deleting the missing observations).

\begin{Shaded}
\begin{Highlighting}[]
\KeywordTok{clear}
\NormalTok{import delimited }\StringTok{"https://www.dropbox.com/scl/fi/ywlbb7duvez2nyk66ojp1/hersdata.csv?rlkey=tmhzlv6ga5zp6uyysnosaqamj\&st=gl7h7ym8\&dl=1"}
\KeywordTok{encode}\NormalTok{  exercise, }\KeywordTok{gen}\NormalTok{(exercise\_r)}
\KeywordTok{encode}\NormalTok{ drinkany , }\KeywordTok{gen}\NormalTok{(drinkany\_r)}

\KeywordTok{drop} \KeywordTok{if}\NormalTok{ diabetes == }\StringTok{"yes"}

\NormalTok{* remove diabetes patients}

\KeywordTok{drop} \KeywordTok{if}\NormalTok{ bmi ==. | drinkany\_r == . }

\NormalTok{* only }\FunctionTok{missing}\NormalTok{ observations }\KeywordTok{in}\NormalTok{ BMI and drinkany}

\KeywordTok{gen}\NormalTok{ cons=1}

\NormalTok{* We will }\KeywordTok{use}\NormalTok{ only 20\% }\KeywordTok{of}\NormalTok{ the }\KeywordTok{data}
\NormalTok{* Due to the licence limitation }\KeywordTok{of} \KeywordTok{STATA}
\NormalTok{* Otherwise you will }\FunctionTok{get}\NormalTok{ an }\KeywordTok{error}\NormalTok{ message}

\KeywordTok{sample}\NormalTok{ 20}


\KeywordTok{mkmat}\NormalTok{ cons exercise\_r age drinkany\_r bmi, }\FunctionTok{matrix}\NormalTok{(X) }
\KeywordTok{mkmat}\NormalTok{ glucose, }\FunctionTok{matrix}\NormalTok{(Y)}
\FunctionTok{matrix}\NormalTok{ XTX =X\textquotesingle{}*X}
\FunctionTok{matrix}\NormalTok{ invXTX = }\FunctionTok{inv}\NormalTok{(XTX)}
\FunctionTok{matrix}\NormalTok{ b=invXTX*X\textquotesingle{}*Y}
\FunctionTok{matrix} \OtherTok{list}\NormalTok{ b }
\end{Highlighting}
\end{Shaded}

\begin{verbatim}
(encoding automatically selected: ISO-8859-1)
(37 vars, 2,763 obs)



(731 observations deleted)

(4 observations deleted)


(1,622 observations deleted)







b[5,1]
               glucose
      cons    72.41736
exercise_r  -2.1271677
       age   .08683906
drinkany_r   2.4667362
       bmi   .64236868
\end{verbatim}

We get the same LS estimates as the ones reported in the Stata output.
You can get the SEs using the following code (that computes first
\(\hat\sigma\))

\begin{Shaded}
\begin{Highlighting}[]
\KeywordTok{clear}
\NormalTok{import delimited }\StringTok{"https://www.dropbox.com/scl/fi/9dtsid3cpziubhhuw9hhy/hers\_subset.csv?rlkey=vainwt6vtbbo2kuqidv0e24v9\&st=k5r7e42w\&dl=1"}
\KeywordTok{encode}\NormalTok{  exercise, }\KeywordTok{gen}\NormalTok{(exercise\_r)}
\KeywordTok{encode}\NormalTok{ drinkany , }\KeywordTok{gen}\NormalTok{(drinkany\_r)}

\KeywordTok{drop} \KeywordTok{if}\NormalTok{ diabetes == }\StringTok{"yes"}

\NormalTok{* remove diabetes patients}
\KeywordTok{drop} \KeywordTok{if}\NormalTok{ bmi ==. | drinkany\_r == . }
\NormalTok{* only }\FunctionTok{missing}\NormalTok{ observations }\KeywordTok{in}\NormalTok{ BMI and drinkany}
\KeywordTok{gen}\NormalTok{ cons=1}

\NormalTok{* We will }\KeywordTok{use}\NormalTok{ only 20\% }\KeywordTok{of}\NormalTok{ the }\KeywordTok{data}
\NormalTok{* Due to the licence limitation }\KeywordTok{of} \KeywordTok{STATA}
\NormalTok{* Otherwise you will }\FunctionTok{get}\NormalTok{ an }\KeywordTok{error}\NormalTok{ message}

\KeywordTok{sample}\NormalTok{ 20}

\KeywordTok{mkmat}\NormalTok{ cons exercise\_r age drinkany\_r bmi, }\FunctionTok{matrix}\NormalTok{(X) }
\KeywordTok{mkmat}\NormalTok{ glucose, }\FunctionTok{matrix}\NormalTok{(Y)}
\FunctionTok{matrix}\NormalTok{ XTX =X\textquotesingle{}*X}
\FunctionTok{matrix}\NormalTok{ invXTX = }\FunctionTok{inv}\NormalTok{(XTX)}
\FunctionTok{matrix}\NormalTok{ b=invXTX*X\textquotesingle{}*Y}
\FunctionTok{matrix} \OtherTok{list}\NormalTok{ b }


\FunctionTok{matrix}\NormalTok{ res=Y{-}X*b}
\FunctionTok{matrix}\NormalTok{ sigma2=res\textquotesingle{}*res/(406{-}5)}

\NormalTok{* df= number }\KeywordTok{of}\NormalTok{ observations(times 20\% }\KeywordTok{of}\NormalTok{ the }\KeywordTok{sample}\NormalTok{) {-} number }\KeywordTok{of}\NormalTok{ parameters=406{-}5}

\FunctionTok{matrix} \OtherTok{list}\NormalTok{ sigma2}

\NormalTok{* it is necessary to transform the }\FunctionTok{matrix}\NormalTok{ (1x1) }\KeywordTok{in}\NormalTok{ a }\FunctionTok{scalar}\NormalTok{ to take the square root}

\FunctionTok{scalar}\NormalTok{ sigma = }\FunctionTok{sqrt}\NormalTok{(sigma2[1,1])}
\KeywordTok{display}\NormalTok{ sigma}
\NormalTok{\#\# (encoding automatically selected: ISO{-}8859{-}1)}
\NormalTok{\#\# (38 vars, 276 }\KeywordTok{obs}\NormalTok{)}
\NormalTok{\#\# }
\NormalTok{\#\# }
\NormalTok{\#\# }
\NormalTok{\#\# (85 observations deleted)}
\NormalTok{\#\# }
\NormalTok{\#\# (0 observations deleted)}
\NormalTok{\#\# }
\NormalTok{\#\# }
\NormalTok{\#\# (153 observations deleted)}
\NormalTok{\#\# }
\NormalTok{\#\# }
\NormalTok{\#\# }
\NormalTok{\#\# }
\NormalTok{\#\# }
\NormalTok{\#\# }
\NormalTok{\#\# }
\NormalTok{\#\# b[5,1]}
\NormalTok{\#\#                glucose}
\NormalTok{\#\#       cons     85.5516}
\NormalTok{\#\# exercise\_r  {-}5.5828251}
\NormalTok{\#\#        age   .30620033}
\NormalTok{\#\# drinkany\_r  {-}4.0001402}
\NormalTok{\#\#        bmi   .13114028}
\NormalTok{\#\# }
\NormalTok{\#\# }
\NormalTok{\#\# }
\NormalTok{\#\# }
\NormalTok{\#\# symmetric sigma2[1,1]}
\NormalTok{\#\#            glucose}
\NormalTok{\#\# glucose  5.7199346}
\NormalTok{\#\# }
\NormalTok{\#\# }
\NormalTok{\#\# 2.3916385}
\end{Highlighting}
\end{Shaded}

This is the estimate reported by Stata as Root MSE. The SEs follow:

\begin{Shaded}
\begin{Highlighting}[]
\KeywordTok{clear}
\NormalTok{import delimited }\StringTok{"https://www.dropbox.com/scl/fi/ywlbb7duvez2nyk66ojp1/hersdata.csv?rlkey=tmhzlv6ga5zp6uyysnosaqamj\&st=gl7h7ym8\&dl=1"}
\KeywordTok{encode}\NormalTok{  exercise, }\KeywordTok{gen}\NormalTok{(exercise\_r)}
\KeywordTok{encode}\NormalTok{ drinkany , }\KeywordTok{gen}\NormalTok{(drinkany\_r)}

\KeywordTok{drop} \KeywordTok{if}\NormalTok{ diabetes == }\StringTok{"yes"}

\NormalTok{* remove diabetes patients}
\KeywordTok{drop} \KeywordTok{if}\NormalTok{ bmi ==. | drinkany\_r == . }
\NormalTok{* only }\FunctionTok{missing}\NormalTok{ observations }\KeywordTok{in}\NormalTok{ BMI and drinkany}
\KeywordTok{gen}\NormalTok{ cons=1}

\NormalTok{* We will }\KeywordTok{use}\NormalTok{ only 20\% }\KeywordTok{of}\NormalTok{ the }\KeywordTok{data}
\NormalTok{* Due to the licence limitation }\KeywordTok{of} \KeywordTok{STATA}
\NormalTok{* Otherwise you will }\FunctionTok{get}\NormalTok{ an }\KeywordTok{error}\NormalTok{ message}

\KeywordTok{sample}\NormalTok{ 20}

\KeywordTok{mkmat}\NormalTok{ cons exercise\_r age drinkany\_r bmi, }\FunctionTok{matrix}\NormalTok{(X) }
\KeywordTok{mkmat}\NormalTok{ glucose, }\FunctionTok{matrix}\NormalTok{(Y)}
\FunctionTok{matrix}\NormalTok{ XTX =X\textquotesingle{}*X}
\FunctionTok{matrix}\NormalTok{ invXTX = }\FunctionTok{inv}\NormalTok{(XTX)}
\FunctionTok{matrix}\NormalTok{ b=invXTX*X\textquotesingle{}*Y}
\FunctionTok{matrix} \OtherTok{list}\NormalTok{ b }

\NormalTok{* Stata code}
\FunctionTok{matrix}\NormalTok{ res=Y{-}X*b}
\FunctionTok{matrix}\NormalTok{ sigma2=res\textquotesingle{}*res/(406*.2{-}5)}
\NormalTok{* df= number }\KeywordTok{of}\NormalTok{ observations {-} number }\KeywordTok{of}\NormalTok{ parameters=406{-}5=401}
\FunctionTok{matrix} \OtherTok{list}\NormalTok{ sigma2}
\NormalTok{* it is necessary to transform the }\FunctionTok{matrix}\NormalTok{ (1x1) }\KeywordTok{in}\NormalTok{ a }\FunctionTok{scalar}\NormalTok{ to take the square root}
\FunctionTok{scalar}\NormalTok{ sigma = }\FunctionTok{sqrt}\NormalTok{(sigma2[1,1])}
\KeywordTok{display}\NormalTok{ sigma}


\NormalTok{* Inverse the }\FunctionTok{matrix}\NormalTok{ X\textquotesingle{}X and extract the diagonal}
\FunctionTok{matrix}\NormalTok{ D=}\FunctionTok{vecdiag}\NormalTok{(invXTX)}
\FunctionTok{matrix} \OtherTok{list}\NormalTok{ D}
\NormalTok{* }\KeywordTok{SE}\NormalTok{(b0) intercept}
\FunctionTok{scalar}\NormalTok{ SE0=}\FunctionTok{sqrt}\NormalTok{(D[1,1])*sigma}
\KeywordTok{display}\NormalTok{ SE0}
\NormalTok{* }\KeywordTok{SE}\NormalTok{(b1) exercise}
\FunctionTok{scalar}\NormalTok{ SE1=}\FunctionTok{sqrt}\NormalTok{(D[1,2])*sigma}
\KeywordTok{display}\NormalTok{ SE1}
\NormalTok{* }\KeywordTok{SE}\NormalTok{(b2) age}
\FunctionTok{scalar}\NormalTok{ SE2=}\FunctionTok{sqrt}\NormalTok{(D[1,3])*sigma}
\KeywordTok{display}\NormalTok{ SE2}
\NormalTok{\#\# (encoding automatically selected: ISO{-}8859{-}1)}
\NormalTok{\#\# (37 vars, 2,763 }\KeywordTok{obs}\NormalTok{)}
\NormalTok{\#\# }
\NormalTok{\#\# }
\NormalTok{\#\# }
\NormalTok{\#\# (731 observations deleted)}
\NormalTok{\#\# }
\NormalTok{\#\# (4 observations deleted)}
\NormalTok{\#\# }
\NormalTok{\#\# }
\NormalTok{\#\# (1,622 observations deleted)}
\NormalTok{\#\# }
\NormalTok{\#\# }
\NormalTok{\#\# }
\NormalTok{\#\# }
\NormalTok{\#\# }
\NormalTok{\#\# }
\NormalTok{\#\# }
\NormalTok{\#\# b[5,1]}
\NormalTok{\#\#                glucose}
\NormalTok{\#\#       cons    77.33811}
\NormalTok{\#\# exercise\_r  {-}2.0892924}
\NormalTok{\#\#        age   .11858697}
\NormalTok{\#\# drinkany\_r   .21326717}
\NormalTok{\#\#        bmi   .51288159}
\NormalTok{\#\# }
\NormalTok{\#\# }
\NormalTok{\#\# }
\NormalTok{\#\# }
\NormalTok{\#\# symmetric sigma2[1,1]}
\NormalTok{\#\#          glucose}
\NormalTok{\#\# glucose  475.952}
\NormalTok{\#\# }
\NormalTok{\#\# }
\NormalTok{\#\# 21.816324}
\NormalTok{\#\# }
\NormalTok{\#\# }
\NormalTok{\#\# }
\NormalTok{\#\# D[1,5]}
\NormalTok{\#\#           cons  exercise\_r         age  drinkany\_r         bmi}
\NormalTok{\#\# r1   .44899007   .01034199    .0000568   .01019343   .00010571}
\NormalTok{\#\# }
\NormalTok{\#\# }
\NormalTok{\#\# 14.618404}
\NormalTok{\#\# }
\NormalTok{\#\# }
\NormalTok{\#\# 2.2186237}
\NormalTok{\#\# }
\NormalTok{\#\# }
\NormalTok{\#\# .16442237}
\end{Highlighting}
\end{Shaded}

And so on for the other coefficients. They match the SEs reported in the
Stata output.

\subsubsection*{Exercise 3}\label{exercise-3}
\addcontentsline{toc}{subsubsection}{Exercise 3}

Solution in Stata

\begin{Shaded}
\begin{Highlighting}[]
\KeywordTok{clear}
\NormalTok{import delimited }\StringTok{"https://www.dropbox.com/scl/fi/ywlbb7duvez2nyk66ojp1/hersdata.csv?rlkey=tmhzlv6ga5zp6uyysnosaqamj\&st=gl7h7ym8\&dl=1"}
\KeywordTok{encode}\NormalTok{  exercise, }\KeywordTok{gen}\NormalTok{(exercise\_r)}
\KeywordTok{encode}\NormalTok{ drinkany , }\KeywordTok{gen}\NormalTok{(drinkany\_r)}

\KeywordTok{drop} \KeywordTok{if}\NormalTok{ diabetes == }\StringTok{"yes"}

\KeywordTok{regress}\NormalTok{ glucose exercise\_r age drinkany\_r bmi }
\NormalTok{* 95\% CI }\KeywordTok{for}\NormalTok{ the }\KeywordTok{mean}\NormalTok{ glucoae }\KeywordTok{for}\NormalTok{ a patient}
\NormalTok{* aged 65, no exercise, no drinking, BMI=29}
\KeywordTok{adjust}\NormalTok{ exercise\_r=0 age=65 drinkany\_r=0 bmi=29, }\KeywordTok{ci}
\NormalTok{\#\# (encoding automatically selected: ISO{-}8859{-}1)}
\NormalTok{\#\# (37 vars, 2,763 }\KeywordTok{obs}\NormalTok{)}
\NormalTok{\#\# }
\NormalTok{\#\# }
\NormalTok{\#\# }
\NormalTok{\#\# (731 observations deleted)}
\NormalTok{\#\# }
\NormalTok{\#\# }
\NormalTok{\#\#       Source |       SS           df       MS      Number }\KeywordTok{of} \KeywordTok{obs}\NormalTok{   =     2,028}
\NormalTok{\#\# {-}{-}{-}{-}{-}{-}{-}{-}{-}{-}{-}{-}{-}+{-}{-}{-}{-}{-}{-}{-}{-}{-}{-}{-}{-}{-}{-}{-}{-}{-}{-}{-}{-}{-}{-}{-}{-}{-}{-}{-}{-}{-}{-}{-}{-}{-}{-}   }\FunctionTok{F}\NormalTok{(4, 2023)      =     39.22}
\NormalTok{\#\#        Model |  13828.8486         4  3457.21214   Prob \textgreater{} }\FunctionTok{F}\NormalTok{        =    0.0000}
\NormalTok{\#\#     Residual |  178319.973     2,023  88.1463042   R{-}squared       =    0.0720}
\NormalTok{\#\# {-}{-}{-}{-}{-}{-}{-}{-}{-}{-}{-}{-}{-}+{-}{-}{-}{-}{-}{-}{-}{-}{-}{-}{-}{-}{-}{-}{-}{-}{-}{-}{-}{-}{-}{-}{-}{-}{-}{-}{-}{-}{-}{-}{-}{-}{-}{-}   Adj R{-}squared   =    0.0701}
\NormalTok{\#\#        Total |  192148.822     2,027  94.7946828   Root MSE        =    9.3886}
\NormalTok{\#\# }
\NormalTok{\#\# {-}{-}{-}{-}{-}{-}{-}{-}{-}{-}{-}{-}{-}{-}{-}{-}{-}{-}{-}{-}{-}{-}{-}{-}{-}{-}{-}{-}{-}{-}{-}{-}{-}{-}{-}{-}{-}{-}{-}{-}{-}{-}{-}{-}{-}{-}{-}{-}{-}{-}{-}{-}{-}{-}{-}{-}{-}{-}{-}{-}{-}{-}{-}{-}{-}{-}{-}{-}{-}{-}{-}{-}{-}{-}{-}{-}{-}{-}}
\NormalTok{\#\#      glucose | Coefficient  Std. err.      t    P\textgreater{}|t|     [95\% conf. interval]}
\NormalTok{\#\# {-}{-}{-}{-}{-}{-}{-}{-}{-}{-}{-}{-}{-}+{-}{-}{-}{-}{-}{-}{-}{-}{-}{-}{-}{-}{-}{-}{-}{-}{-}{-}{-}{-}{-}{-}{-}{-}{-}{-}{-}{-}{-}{-}{-}{-}{-}{-}{-}{-}{-}{-}{-}{-}{-}{-}{-}{-}{-}{-}{-}{-}{-}{-}{-}{-}{-}{-}{-}{-}{-}{-}{-}{-}{-}{-}{-}{-}}
\NormalTok{\#\#   exercise\_r |   {-}.950441     .42873    {-}2.22   0.027    {-}1.791239   {-}.1096426}
\NormalTok{\#\#          age |   .0635495   .0313911     2.02   0.043     .0019872    .1251118}
\NormalTok{\#\#   drinkany\_r |   .6802641   .4219569     1.61   0.107    {-}.1472514     1.50778}
\NormalTok{\#\#          bmi |    .489242   .0415528    11.77   0.000     .4077512    .5707328}
\NormalTok{\#\#        }\DataTypeTok{\_cons}\NormalTok{ |   79.23257   2.788671    28.41   0.000      73.7636    84.70154}
\NormalTok{\#\# {-}{-}{-}{-}{-}{-}{-}{-}{-}{-}{-}{-}{-}{-}{-}{-}{-}{-}{-}{-}{-}{-}{-}{-}{-}{-}{-}{-}{-}{-}{-}{-}{-}{-}{-}{-}{-}{-}{-}{-}{-}{-}{-}{-}{-}{-}{-}{-}{-}{-}{-}{-}{-}{-}{-}{-}{-}{-}{-}{-}{-}{-}{-}{-}{-}{-}{-}{-}{-}{-}{-}{-}{-}{-}{-}{-}{-}{-}}
\NormalTok{\#\# }
\NormalTok{\#\# }
\NormalTok{\#\# {-}{-}{-}{-}{-}{-}{-}{-}{-}{-}{-}{-}{-}{-}{-}{-}{-}{-}{-}{-}{-}{-}{-}{-}{-}{-}{-}{-}{-}{-}{-}{-}{-}{-}{-}{-}{-}{-}{-}{-}{-}{-}{-}{-}{-}{-}{-}{-}{-}{-}{-}{-}{-}{-}{-}{-}{-}{-}{-}{-}{-}{-}{-}{-}{-}{-}{-}{-}{-}{-}{-}{-}{-}{-}{-}{-}{-}{-}{-}{-}{-}{-}{-}{-}{-}{-}{-}{-}{-}{-}{-}{-}{-}{-}{-}{-}{-}{-}{-}{-}{-}{-}{-}{-}{-}{-}{-}{-}{-}{-}{-}{-}{-}{-}{-}{-}{-}{-}{-}{-}{-}{-}{-}{-}{-}{-}{-}}
\NormalTok{\#\#      Dependent }\KeywordTok{variable}\NormalTok{: glucose     Command: }\KeywordTok{regress}
\NormalTok{\#\# Covariates }\KeywordTok{set}\NormalTok{ to }\OtherTok{value}\NormalTok{: exercise\_r = 0, age = 65, drinkany\_r = 0, bmi = 29}
\NormalTok{\#\# {-}{-}{-}{-}{-}{-}{-}{-}{-}{-}{-}{-}{-}{-}{-}{-}{-}{-}{-}{-}{-}{-}{-}{-}{-}{-}{-}{-}{-}{-}{-}{-}{-}{-}{-}{-}{-}{-}{-}{-}{-}{-}{-}{-}{-}{-}{-}{-}{-}{-}{-}{-}{-}{-}{-}{-}{-}{-}{-}{-}{-}{-}{-}{-}{-}{-}{-}{-}{-}{-}{-}{-}{-}{-}{-}{-}{-}{-}{-}{-}{-}{-}{-}{-}{-}{-}{-}{-}{-}{-}{-}{-}{-}{-}{-}{-}{-}{-}{-}{-}{-}{-}{-}{-}{-}{-}{-}{-}{-}{-}{-}{-}{-}{-}{-}{-}{-}{-}{-}{-}{-}{-}{-}{-}{-}{-}{-}}
\NormalTok{\#\# }
\NormalTok{\#\# {-}{-}{-}{-}{-}{-}{-}{-}{-}{-}{-}{-}{-}{-}{-}{-}{-}{-}{-}{-}{-}{-}{-}{-}{-}{-}{-}{-}{-}{-}{-}{-}{-}{-}{-}{-}{-}{-}{-}{-}{-}{-}{-}{-}{-}{-}}
\NormalTok{\#\#       All |         }\KeywordTok{xb}\NormalTok{          lb          ub}
\NormalTok{\#\# {-}{-}{-}{-}{-}{-}{-}{-}{-}{-}+{-}{-}{-}{-}{-}{-}{-}{-}{-}{-}{-}{-}{-}{-}{-}{-}{-}{-}{-}{-}{-}{-}{-}{-}{-}{-}{-}{-}{-}{-}{-}{-}{-}{-}{-}}
\NormalTok{\#\#           |    97.5513    [95.8247    99.2779]}
\NormalTok{\#\# {-}{-}{-}{-}{-}{-}{-}{-}{-}{-}{-}{-}{-}{-}{-}{-}{-}{-}{-}{-}{-}{-}{-}{-}{-}{-}{-}{-}{-}{-}{-}{-}{-}{-}{-}{-}{-}{-}{-}{-}{-}{-}{-}{-}{-}{-}}
\NormalTok{\#\#      Key:  }\KeywordTok{xb}\NormalTok{         =  Linear Prediction}
\NormalTok{\#\#            [lb , ub]  =  [95\% Confidence Interval]}
\end{Highlighting}
\end{Shaded}

Algebraically it follows from the few lines below (assuming we kept the
previous quantities and estimates):

\begin{Shaded}
\begin{Highlighting}[]
\KeywordTok{clear}
\NormalTok{import delimited }\StringTok{"https://www.dropbox.com/scl/fi/ywlbb7duvez2nyk66ojp1/hersdata.csv?rlkey=tmhzlv6ga5zp6uyysnosaqamj\&st=gl7h7ym8\&dl=1"}
\KeywordTok{encode}\NormalTok{  exercise, }\KeywordTok{gen}\NormalTok{(exercise\_r)}
\KeywordTok{encode}\NormalTok{ drinkany , }\KeywordTok{gen}\NormalTok{(drinkany\_r)}

\KeywordTok{drop} \KeywordTok{if}\NormalTok{ diabetes == }\StringTok{"yes"}

\NormalTok{* remove diabetes patients}
\KeywordTok{drop} \KeywordTok{if}\NormalTok{ bmi ==. | drinkany\_r == . }
\NormalTok{* only }\FunctionTok{missing}\NormalTok{ observations }\KeywordTok{in}\NormalTok{ BMI and drinkany}
\KeywordTok{gen}\NormalTok{ cons=1}


\NormalTok{* We will }\KeywordTok{use}\NormalTok{ only 20\% }\KeywordTok{of}\NormalTok{ the }\KeywordTok{data}
\NormalTok{* Due to the licence limitation }\KeywordTok{of} \KeywordTok{STATA}
\NormalTok{* Otherwise you will }\FunctionTok{get}\NormalTok{ an }\KeywordTok{error}\NormalTok{ message}

\KeywordTok{sample}\NormalTok{ 20}


\KeywordTok{mkmat}\NormalTok{ cons exercise\_r age drinkany\_r bmi, }\FunctionTok{matrix}\NormalTok{(X) }
\KeywordTok{mkmat}\NormalTok{ glucose, }\FunctionTok{matrix}\NormalTok{(Y)}
\FunctionTok{matrix}\NormalTok{ XTX =X\textquotesingle{}*X}
\FunctionTok{matrix}\NormalTok{ invXTX = }\FunctionTok{inv}\NormalTok{(XTX)}
\FunctionTok{matrix}\NormalTok{ b=invXTX*X\textquotesingle{}*Y}
\FunctionTok{matrix} \OtherTok{list}\NormalTok{ b }



\NormalTok{* Stata code}
\FunctionTok{matrix}\NormalTok{ res=Y{-}X*b}
\FunctionTok{matrix}\NormalTok{ sigma2=res\textquotesingle{}*res/(406{-}5)}
\NormalTok{* df= number }\KeywordTok{of}\NormalTok{ observations {-} number }\KeywordTok{of}\NormalTok{ parameters=406{-}5=401}
\FunctionTok{matrix} \OtherTok{list}\NormalTok{ sigma2}
\NormalTok{* it is necessary to transform the }\FunctionTok{matrix}\NormalTok{ (1x1) }\KeywordTok{in}\NormalTok{ a }\FunctionTok{scalar}\NormalTok{ to take the square root}
\FunctionTok{scalar}\NormalTok{ sigma = }\FunctionTok{sqrt}\NormalTok{(sigma2[1,1])}
\KeywordTok{display}\NormalTok{ sigma}


\NormalTok{* Inverse the }\FunctionTok{matrix}\NormalTok{ X\textquotesingle{}X and extract the diagonal}
\FunctionTok{matrix}\NormalTok{ D=}\FunctionTok{vecdiag}\NormalTok{(invXTX)}
\FunctionTok{matrix} \OtherTok{list}\NormalTok{ D}
\NormalTok{*}\KeywordTok{SE}\NormalTok{(b0) intercept}
\FunctionTok{scalar}\NormalTok{ SE0=}\FunctionTok{sqrt}\NormalTok{(D[1,1])*sigma}
\KeywordTok{display}\NormalTok{ SE0}
\NormalTok{*}\KeywordTok{SE}\NormalTok{(b1) exercise}
\FunctionTok{scalar}\NormalTok{ SE1=}\FunctionTok{sqrt}\NormalTok{(D[1,2])*sigma}
\KeywordTok{display}\NormalTok{ SE1}
\NormalTok{*}\KeywordTok{SE}\NormalTok{(b2) age}
\FunctionTok{scalar}\NormalTok{ SE2=}\FunctionTok{sqrt}\NormalTok{(D[1,3])*sigma}
\KeywordTok{display}\NormalTok{ SE2}



\FunctionTok{matrix}\NormalTok{ profile=(1,0,65,0,29) }
\FunctionTok{matrix}\NormalTok{ E=profile*invXTX*profile\textquotesingle{}}
\FunctionTok{scalar} \KeywordTok{SE}\NormalTok{=sigma*}\FunctionTok{sqrt}\NormalTok{(E[1,1])}
\KeywordTok{display} \KeywordTok{SE}
\FunctionTok{matrix}\NormalTok{ pred=profile*b}
\KeywordTok{display}\NormalTok{ pred[1,1]    }
\FunctionTok{matrix} \OtherTok{list}\NormalTok{ pred }
\FunctionTok{matrix} \FunctionTok{lower}\NormalTok{=pred{-}1.96*}\KeywordTok{SE}
\FunctionTok{matrix} \FunctionTok{upper}\NormalTok{=pred+1.96*}\KeywordTok{SE}
\FunctionTok{matrix} \OtherTok{list} \FunctionTok{lower}
\FunctionTok{matrix} \OtherTok{list} \FunctionTok{upper} 
\NormalTok{\#\# (encoding automatically selected: ISO{-}8859{-}1)}
\NormalTok{\#\# (37 vars, 2,763 }\KeywordTok{obs}\NormalTok{)}
\NormalTok{\#\# }
\NormalTok{\#\# }
\NormalTok{\#\# }
\NormalTok{\#\# (731 observations deleted)}
\NormalTok{\#\# }
\NormalTok{\#\# (4 observations deleted)}
\NormalTok{\#\# }
\NormalTok{\#\# }
\NormalTok{\#\# (1,622 observations deleted)}
\NormalTok{\#\# }
\NormalTok{\#\# }
\NormalTok{\#\# }
\NormalTok{\#\# }
\NormalTok{\#\# }
\NormalTok{\#\# }
\NormalTok{\#\# }
\NormalTok{\#\# b[5,1]}
\NormalTok{\#\#                glucose}
\NormalTok{\#\#       cons   80.554422}
\NormalTok{\#\# exercise\_r  {-}1.1625011}
\NormalTok{\#\#        age    .0820942}
\NormalTok{\#\# drinkany\_r   1.4372651}
\NormalTok{\#\#        bmi   .37887655}
\NormalTok{\#\# }
\NormalTok{\#\# }
\NormalTok{\#\# }
\NormalTok{\#\# }
\NormalTok{\#\# symmetric sigma2[1,1]}
\NormalTok{\#\#            glucose}
\NormalTok{\#\# glucose  91.055561}
\NormalTok{\#\# }
\NormalTok{\#\# }
\NormalTok{\#\# 9.5423038}
\NormalTok{\#\# }
\NormalTok{\#\# }
\NormalTok{\#\# }
\NormalTok{\#\# D[1,5]}
\NormalTok{\#\#           cons  exercise\_r         age  drinkany\_r         bmi}
\NormalTok{\#\# r1    .4530393   .01082784   .00005606   .01020371   .00010226}
\NormalTok{\#\# }
\NormalTok{\#\# }
\NormalTok{\#\# 6.4227523}
\NormalTok{\#\# }
\NormalTok{\#\# }
\NormalTok{\#\# .99294249}
\NormalTok{\#\# }
\NormalTok{\#\# }
\NormalTok{\#\# .07144555}
\NormalTok{\#\# }
\NormalTok{\#\# }
\NormalTok{\#\# }
\NormalTok{\#\# }
\NormalTok{\#\# 1.96577}
\NormalTok{\#\# }
\NormalTok{\#\# }
\NormalTok{\#\# 96.877965}
\NormalTok{\#\# }
\NormalTok{\#\# }
\NormalTok{\#\# symmetric pred[1,1]}
\NormalTok{\#\#       glucose}
\NormalTok{\#\# r1  96.877965}
\NormalTok{\#\# }
\NormalTok{\#\# }
\NormalTok{\#\# }
\NormalTok{\#\# }
\NormalTok{\#\# symmetric }\FunctionTok{lower}\NormalTok{[1,1]}
\NormalTok{\#\#            c1}
\NormalTok{\#\# r1  93.025056}
\NormalTok{\#\# }
\NormalTok{\#\# }
\NormalTok{\#\# symmetric }\FunctionTok{upper}\NormalTok{[1,1]}
\NormalTok{\#\#            c1}
\NormalTok{\#\# r1  100.73087}
\end{Highlighting}
\end{Shaded}

The same values (up to rounding) are obtained once again. We used here
1.96 since the sample is large but the exact quantile from the
appropriate \(t\)-distribution should be used in smaller samples.




\end{document}
