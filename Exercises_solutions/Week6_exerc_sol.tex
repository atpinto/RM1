% Options for packages loaded elsewhere
\PassOptionsToPackage{unicode}{hyperref}
\PassOptionsToPackage{hyphens}{url}
\PassOptionsToPackage{dvipsnames,svgnames,x11names}{xcolor}
%
\documentclass[
  letterpaper,
  DIV=11,
  numbers=noendperiod]{scrreprt}

\usepackage{amsmath,amssymb}
\usepackage{iftex}
\ifPDFTeX
  \usepackage[T1]{fontenc}
  \usepackage[utf8]{inputenc}
  \usepackage{textcomp} % provide euro and other symbols
\else % if luatex or xetex
  \usepackage{unicode-math}
  \defaultfontfeatures{Scale=MatchLowercase}
  \defaultfontfeatures[\rmfamily]{Ligatures=TeX,Scale=1}
\fi
\usepackage{lmodern}
\ifPDFTeX\else  
    % xetex/luatex font selection
\fi
% Use upquote if available, for straight quotes in verbatim environments
\IfFileExists{upquote.sty}{\usepackage{upquote}}{}
\IfFileExists{microtype.sty}{% use microtype if available
  \usepackage[]{microtype}
  \UseMicrotypeSet[protrusion]{basicmath} % disable protrusion for tt fonts
}{}
\makeatletter
\@ifundefined{KOMAClassName}{% if non-KOMA class
  \IfFileExists{parskip.sty}{%
    \usepackage{parskip}
  }{% else
    \setlength{\parindent}{0pt}
    \setlength{\parskip}{6pt plus 2pt minus 1pt}}
}{% if KOMA class
  \KOMAoptions{parskip=half}}
\makeatother
\usepackage{xcolor}
\setlength{\emergencystretch}{3em} % prevent overfull lines
\setcounter{secnumdepth}{-\maxdimen} % remove section numbering
% Make \paragraph and \subparagraph free-standing
\makeatletter
\ifx\paragraph\undefined\else
  \let\oldparagraph\paragraph
  \renewcommand{\paragraph}{
    \@ifstar
      \xxxParagraphStar
      \xxxParagraphNoStar
  }
  \newcommand{\xxxParagraphStar}[1]{\oldparagraph*{#1}\mbox{}}
  \newcommand{\xxxParagraphNoStar}[1]{\oldparagraph{#1}\mbox{}}
\fi
\ifx\subparagraph\undefined\else
  \let\oldsubparagraph\subparagraph
  \renewcommand{\subparagraph}{
    \@ifstar
      \xxxSubParagraphStar
      \xxxSubParagraphNoStar
  }
  \newcommand{\xxxSubParagraphStar}[1]{\oldsubparagraph*{#1}\mbox{}}
  \newcommand{\xxxSubParagraphNoStar}[1]{\oldsubparagraph{#1}\mbox{}}
\fi
\makeatother

\usepackage{color}
\usepackage{fancyvrb}
\newcommand{\VerbBar}{|}
\newcommand{\VERB}{\Verb[commandchars=\\\{\}]}
\DefineVerbatimEnvironment{Highlighting}{Verbatim}{commandchars=\\\{\}}
% Add ',fontsize=\small' for more characters per line
\usepackage{framed}
\definecolor{shadecolor}{RGB}{241,243,245}
\newenvironment{Shaded}{\begin{snugshade}}{\end{snugshade}}
\newcommand{\AlertTok}[1]{\textcolor[rgb]{0.68,0.00,0.00}{#1}}
\newcommand{\AnnotationTok}[1]{\textcolor[rgb]{0.37,0.37,0.37}{#1}}
\newcommand{\AttributeTok}[1]{\textcolor[rgb]{0.40,0.45,0.13}{#1}}
\newcommand{\BaseNTok}[1]{\textcolor[rgb]{0.68,0.00,0.00}{#1}}
\newcommand{\BuiltInTok}[1]{\textcolor[rgb]{0.00,0.23,0.31}{#1}}
\newcommand{\CharTok}[1]{\textcolor[rgb]{0.13,0.47,0.30}{#1}}
\newcommand{\CommentTok}[1]{\textcolor[rgb]{0.37,0.37,0.37}{#1}}
\newcommand{\CommentVarTok}[1]{\textcolor[rgb]{0.37,0.37,0.37}{\textit{#1}}}
\newcommand{\ConstantTok}[1]{\textcolor[rgb]{0.56,0.35,0.01}{#1}}
\newcommand{\ControlFlowTok}[1]{\textcolor[rgb]{0.00,0.23,0.31}{\textbf{#1}}}
\newcommand{\DataTypeTok}[1]{\textcolor[rgb]{0.68,0.00,0.00}{#1}}
\newcommand{\DecValTok}[1]{\textcolor[rgb]{0.68,0.00,0.00}{#1}}
\newcommand{\DocumentationTok}[1]{\textcolor[rgb]{0.37,0.37,0.37}{\textit{#1}}}
\newcommand{\ErrorTok}[1]{\textcolor[rgb]{0.68,0.00,0.00}{#1}}
\newcommand{\ExtensionTok}[1]{\textcolor[rgb]{0.00,0.23,0.31}{#1}}
\newcommand{\FloatTok}[1]{\textcolor[rgb]{0.68,0.00,0.00}{#1}}
\newcommand{\FunctionTok}[1]{\textcolor[rgb]{0.28,0.35,0.67}{#1}}
\newcommand{\ImportTok}[1]{\textcolor[rgb]{0.00,0.46,0.62}{#1}}
\newcommand{\InformationTok}[1]{\textcolor[rgb]{0.37,0.37,0.37}{#1}}
\newcommand{\KeywordTok}[1]{\textcolor[rgb]{0.00,0.23,0.31}{\textbf{#1}}}
\newcommand{\NormalTok}[1]{\textcolor[rgb]{0.00,0.23,0.31}{#1}}
\newcommand{\OperatorTok}[1]{\textcolor[rgb]{0.37,0.37,0.37}{#1}}
\newcommand{\OtherTok}[1]{\textcolor[rgb]{0.00,0.23,0.31}{#1}}
\newcommand{\PreprocessorTok}[1]{\textcolor[rgb]{0.68,0.00,0.00}{#1}}
\newcommand{\RegionMarkerTok}[1]{\textcolor[rgb]{0.00,0.23,0.31}{#1}}
\newcommand{\SpecialCharTok}[1]{\textcolor[rgb]{0.37,0.37,0.37}{#1}}
\newcommand{\SpecialStringTok}[1]{\textcolor[rgb]{0.13,0.47,0.30}{#1}}
\newcommand{\StringTok}[1]{\textcolor[rgb]{0.13,0.47,0.30}{#1}}
\newcommand{\VariableTok}[1]{\textcolor[rgb]{0.07,0.07,0.07}{#1}}
\newcommand{\VerbatimStringTok}[1]{\textcolor[rgb]{0.13,0.47,0.30}{#1}}
\newcommand{\WarningTok}[1]{\textcolor[rgb]{0.37,0.37,0.37}{\textit{#1}}}

\providecommand{\tightlist}{%
  \setlength{\itemsep}{0pt}\setlength{\parskip}{0pt}}\usepackage{longtable,booktabs,array}
\usepackage{calc} % for calculating minipage widths
% Correct order of tables after \paragraph or \subparagraph
\usepackage{etoolbox}
\makeatletter
\patchcmd\longtable{\par}{\if@noskipsec\mbox{}\fi\par}{}{}
\makeatother
% Allow footnotes in longtable head/foot
\IfFileExists{footnotehyper.sty}{\usepackage{footnotehyper}}{\usepackage{footnote}}
\makesavenoteenv{longtable}
\usepackage{graphicx}
\makeatletter
\newsavebox\pandoc@box
\newcommand*\pandocbounded[1]{% scales image to fit in text height/width
  \sbox\pandoc@box{#1}%
  \Gscale@div\@tempa{\textheight}{\dimexpr\ht\pandoc@box+\dp\pandoc@box\relax}%
  \Gscale@div\@tempb{\linewidth}{\wd\pandoc@box}%
  \ifdim\@tempb\p@<\@tempa\p@\let\@tempa\@tempb\fi% select the smaller of both
  \ifdim\@tempa\p@<\p@\scalebox{\@tempa}{\usebox\pandoc@box}%
  \else\usebox{\pandoc@box}%
  \fi%
}
% Set default figure placement to htbp
\def\fps@figure{htbp}
\makeatother

\KOMAoption{captions}{tableheading}
\makeatletter
\@ifpackageloaded{caption}{}{\usepackage{caption}}
\AtBeginDocument{%
\ifdefined\contentsname
  \renewcommand*\contentsname{Table of contents}
\else
  \newcommand\contentsname{Table of contents}
\fi
\ifdefined\listfigurename
  \renewcommand*\listfigurename{List of Figures}
\else
  \newcommand\listfigurename{List of Figures}
\fi
\ifdefined\listtablename
  \renewcommand*\listtablename{List of Tables}
\else
  \newcommand\listtablename{List of Tables}
\fi
\ifdefined\figurename
  \renewcommand*\figurename{Figure}
\else
  \newcommand\figurename{Figure}
\fi
\ifdefined\tablename
  \renewcommand*\tablename{Table}
\else
  \newcommand\tablename{Table}
\fi
}
\@ifpackageloaded{float}{}{\usepackage{float}}
\floatstyle{ruled}
\@ifundefined{c@chapter}{\newfloat{codelisting}{h}{lop}}{\newfloat{codelisting}{h}{lop}[chapter]}
\floatname{codelisting}{Listing}
\newcommand*\listoflistings{\listof{codelisting}{List of Listings}}
\makeatother
\makeatletter
\makeatother
\makeatletter
\@ifpackageloaded{caption}{}{\usepackage{caption}}
\@ifpackageloaded{subcaption}{}{\usepackage{subcaption}}
\makeatother

\usepackage{bookmark}

\IfFileExists{xurl.sty}{\usepackage{xurl}}{} % add URL line breaks if available
\urlstyle{same} % disable monospaced font for URLs
\hypersetup{
  pdftitle={Week6-Exercises-Solutions},
  colorlinks=true,
  linkcolor={blue},
  filecolor={Maroon},
  citecolor={Blue},
  urlcolor={Blue},
  pdfcreator={LaTeX via pandoc}}


\title{Week6-Exercises-Solutions}
\author{}
\date{}

\begin{document}
\maketitle


\chapter{Exercise solutions}\label{exercise-solutions}

\subsection*{Week 6}\label{week-6}
\addcontentsline{toc}{subsection}{Week 6}

There are a few issues with this dataset, namely the potential
non-linearity that can be corrected with a log-transformation of fev1.
However we will put this issue aside so that we can focus on
collinearity.

Carrying out the regression, and calculating the variance inflation
factors (VIF) gives some very high VIFs (e.g.~armspan, ulna length,
forearm length, and height), and some moderately high VIFs with age and
weight. This makes a lot of sense, armspan, ulna length and forearm
lengths are just different measures of ``arm length''. We would also
expect children that are taller to have longer arms, weight more, and be
older children. So lots of collinearity issues here.

The VIF is a measure of how much each covariate is associated with the
other covariates in the model. To compute the VIF for the covariate age,
for example, we could run a regression for age using all the other
covariates:

\[\overline{age} =\alpha_0 + \alpha_1 wt + \alpha_2 armsp + \alpha_3 ulna + \alpha_4 farm\]
And the \(VIF(age) =\frac{1}{1-R^2}\), where \(R^2\) is computed from
the model above.

We will first address the similar ``arm length'' measurements by running
four models, one model for each armlength (with the other two removed).

These three models certainly elimate some of the collinearity, but there
does still seem to be bad collinearity between height and the remaining
armlength variable, and between age, weight (and potentially height).
Let's next explore removing either height, or the armlenth variable to
see how well this addresses the issue.

This seems to address the issue suitably. Although there is still
collinearity between the remaining variables, all still remain strongly
associated with the outcome (indicated by small p-values), and so
despite their correlations, they are making independent contributions to
explaining the variance in FEV1. You may also have noticed how removal
of collinear variables has reduced the standard errors in the remaining
variables considerably - a good thing.

Finally we must decide which analysis is most suitable for our research
purpose. The final four models all are reasonably similar from a
statistical perspective, with only a 2\% difference observed in
R-squared values (between 82\% and 84\% of variance explained across the
models). Given this similarity, instead of choosing the model with the
highest R-squared value, we may wish to consider which of the three
measurements is most easily captured with low measurement error. Here we
can rule out ulna length, as this is most likely the most difficult to
measure accurately and non-invasively. Height, may be an obvious choice
here - but it would depend on your sample. If your sample is likely to
contain children who are unwell (lying in bed), or are in wheelchairs,
height would be difficult to measure. Similarly, armspan may be
difficult to capture for children with arm injuries. Given these
factors, forearm length seems an appropriate choice for this model, and
has only a marginally smaller R-squared value than height.

Stata code ::: \{.cell\}

\begin{Shaded}
\begin{Highlighting}[]
\NormalTok{import delimited }\StringTok{"https://www.dropbox.com/s/2cs0z39v2ekni81/lungfun.csv?dl=1"}

\NormalTok{* Initial full }\KeywordTok{model}
\KeywordTok{reg}\NormalTok{ fev1 age wt ht armsp ulna farm}
\KeywordTok{vif}

\NormalTok{* Models with only 1 arm }\FunctionTok{length}\NormalTok{ measurement}
\KeywordTok{reg}\NormalTok{ fev1 age wt ht armsp}
\KeywordTok{vif}
\KeywordTok{reg}\NormalTok{ fev1 age wt ht ulna}
\KeywordTok{vif}
\KeywordTok{reg}\NormalTok{ fev1 age wt ht farm}
\KeywordTok{vif}

\NormalTok{* Models with only 1 arm }\FunctionTok{length}\NormalTok{ measurement }\KeywordTok{or}\NormalTok{ with height (but }\KeywordTok{not}\NormalTok{ both)}
\KeywordTok{reg}\NormalTok{ fev1 age wt ht}
\KeywordTok{vif}
\KeywordTok{reg}\NormalTok{ fev1 age wt armsp}
\KeywordTok{vif}
\KeywordTok{reg}\NormalTok{ fev1 age wt ulna}
\KeywordTok{vif}
\KeywordTok{reg}\NormalTok{ fev1 age wt farm}
\KeywordTok{vif}
\end{Highlighting}
\end{Shaded}

:::

R code ::: \{.cell\}

\begin{Shaded}
\begin{Highlighting}[]
\FunctionTok{library}\NormalTok{(car)}
\NormalTok{lungfun }\OtherTok{\textless{}{-}} \FunctionTok{read.csv}\NormalTok{(}\StringTok{"https://www.dropbox.com/scl/fi/eee36d1h85s8gnejdqykx/lungfun.csv?rlkey=5sd9pdkhc0r9816iiviqljqtx\&st=607s9yxt\&dl=1"}\NormalTok{)}

\CommentTok{\# Initial full model}
\NormalTok{lungfun.lm1 }\OtherTok{\textless{}{-}} \FunctionTok{lm}\NormalTok{(fev1 }\SpecialCharTok{\textasciitilde{}}\NormalTok{ age }\SpecialCharTok{+}\NormalTok{ wt }\SpecialCharTok{+}\NormalTok{ ht }\SpecialCharTok{+}\NormalTok{ armsp }\SpecialCharTok{+}\NormalTok{ ulna }\SpecialCharTok{+}\NormalTok{ farm, }\AttributeTok{data=}\NormalTok{lungfun)}
\FunctionTok{summary}\NormalTok{(lungfun.lm1)}
\FunctionTok{vif}\NormalTok{(lungfun.lm1)}

\CommentTok{\# Models with only 1 arm length measurement}
\NormalTok{lungfun.lm3 }\OtherTok{\textless{}{-}} \FunctionTok{lm}\NormalTok{(fev1 }\SpecialCharTok{\textasciitilde{}}\NormalTok{ age }\SpecialCharTok{+}\NormalTok{ wt }\SpecialCharTok{+}\NormalTok{ ht }\SpecialCharTok{+}\NormalTok{ armsp              , }\AttributeTok{data=}\NormalTok{lungfun)}
\NormalTok{lungfun.lm2 }\OtherTok{\textless{}{-}} \FunctionTok{lm}\NormalTok{(fev1 }\SpecialCharTok{\textasciitilde{}}\NormalTok{ age }\SpecialCharTok{+}\NormalTok{ wt }\SpecialCharTok{+}\NormalTok{ ht }\SpecialCharTok{+}\NormalTok{         ulna       , }\AttributeTok{data=}\NormalTok{lungfun)}
\NormalTok{lungfun.lm4 }\OtherTok{\textless{}{-}} \FunctionTok{lm}\NormalTok{(fev1 }\SpecialCharTok{\textasciitilde{}}\NormalTok{ age }\SpecialCharTok{+}\NormalTok{ wt }\SpecialCharTok{+}\NormalTok{ ht }\SpecialCharTok{+}\NormalTok{               farm , }\AttributeTok{data=}\NormalTok{lungfun)}
\FunctionTok{vif}\NormalTok{(lungfun.lm2)}
\FunctionTok{vif}\NormalTok{(lungfun.lm3)}
\FunctionTok{vif}\NormalTok{(lungfun.lm4)}
\FunctionTok{summary}\NormalTok{(lungfun.lm2)}
\FunctionTok{summary}\NormalTok{(lungfun.lm3)}
\FunctionTok{summary}\NormalTok{(lungfun.lm4)}

\CommentTok{\# Models with only 1 arm length measurement or with height (but not both)}
\NormalTok{lungfun.lm5 }\OtherTok{\textless{}{-}} \FunctionTok{lm}\NormalTok{(fev1 }\SpecialCharTok{\textasciitilde{}}\NormalTok{ age }\SpecialCharTok{+}\NormalTok{ wt }\SpecialCharTok{+}\NormalTok{ ht, }\AttributeTok{data=}\NormalTok{lungfun)}
\NormalTok{lungfun.lm6 }\OtherTok{\textless{}{-}} \FunctionTok{lm}\NormalTok{(fev1 }\SpecialCharTok{\textasciitilde{}}\NormalTok{ age }\SpecialCharTok{+}\NormalTok{ wt }\SpecialCharTok{+}\NormalTok{ armsp, }\AttributeTok{data=}\NormalTok{lungfun)}
\NormalTok{lungfun.lm7 }\OtherTok{\textless{}{-}} \FunctionTok{lm}\NormalTok{(fev1 }\SpecialCharTok{\textasciitilde{}}\NormalTok{ age }\SpecialCharTok{+}\NormalTok{ wt }\SpecialCharTok{+}\NormalTok{ ulna, }\AttributeTok{data=}\NormalTok{lungfun)}
\NormalTok{lungfun.lm8 }\OtherTok{\textless{}{-}} \FunctionTok{lm}\NormalTok{(fev1 }\SpecialCharTok{\textasciitilde{}}\NormalTok{ age }\SpecialCharTok{+}\NormalTok{ wt }\SpecialCharTok{+}\NormalTok{ farm, }\AttributeTok{data=}\NormalTok{lungfun)}
\FunctionTok{vif}\NormalTok{(lungfun.lm5)}
\FunctionTok{vif}\NormalTok{(lungfun.lm6)}
\FunctionTok{vif}\NormalTok{(lungfun.lm7)}
\FunctionTok{vif}\NormalTok{(lungfun.lm8)}
\FunctionTok{summary}\NormalTok{(lungfun.lm5)}
\FunctionTok{summary}\NormalTok{(lungfun.lm6)}
\FunctionTok{summary}\NormalTok{(lungfun.lm7)}
\FunctionTok{summary}\NormalTok{(lungfun.lm8)}
\end{Highlighting}
\end{Shaded}

:::




\end{document}
